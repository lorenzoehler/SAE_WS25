\documentclass[25pt, a0paper, portrait]{tikzposter}

% Bibliographie-Setup
\usepackage[utf8]{inputenc}
\usepackage[ngerman]{babel}
\usepackage{biblatex}
% Falls du eine .bib Datei hast, lade sie hier:
% \addbibresource{references.bib} 

\usepackage{xcolor}
\usepackage{graphicx}
\usepackage{amsmath}
\usepackage{booktabs}

% Uni Bamberg Style laden
\usepackage{Unistyle}
\usetitlestyle{Unistyle}
\useblockstyle{Unistyle}

% Header-Daten
\title{Vergleich der Fay-Herriot (FH) und Battese-Harter-Fuller (BHF) Modelle}
\author{[Eure Namen]}
\institute{Lehrstuhl für Statistik / SAE Lecture -- Simulationsstudie Bolivien}
\titlegraphic{\includegraphics[width=0.12\textwidth]{images/Logo_weiß.pdf}}

\begin{document}

\maketitle

%%%%%%%%%%%%%%%%%%%%%%%%%%%%%%%%%%%%%%%%%%%%%%%%%%%%%%%%%%%%
\block{1. Einleitung \& Zielsetzung}
{
    In der regionalen Planung (Small Area Estimation) sind direkte Schätzer bei kleinen Stichprobenumfängen ($n=20$) oft unzuverlässig. 
    Diese Studie nutzt den \textbf{bolivianischen Zensus 2024} als „Grundgesamtheit“, um die Präzision von zwei Modellansätzen zu validieren:
    \begin{itemize}
        \item \textbf{Fay-Herriot (FH):} Ein Area-Level Modell, das aggregierte Daten nutzt.
        \item \textbf{Battese-Harter-Fuller (BHF):} Ein Unit-Level Modell, das Informationen auf Individualebene nutzt.
    \end{itemize}
    Das Ziel ist die Minimierung des \textbf{Mean Squared Error (MSE)} für die Schätzung der Bildungsjahre (\textit{aestudio}) in 113 Provinzen.
}   

%%%%%%%%%%%%%%%%%%%%%%%%%%%%%%%%%%%%%%%%%%%%%%%%%%%%%%%%%%%%%%%
\begin{columns}
    \column{0.5}
    \block{2. Daten \& Simulationsdesign}{
        \textbf{Datenbasis:} Vollständiger Zensus Boliviens (Haushalts- \& Personendaten). Filter: Alter $\ge 18$ Jahre. \\
        \textbf{Simulation:} 
        \begin{itemize}
            \item 200 unabhängige Stichprobenziehungen (Monte-Carlo).
            \item Stratifizierte Auswahl: Festgelegte Stichprobe von \textbf{$n=20$ pro Provinz}.
            \item Da die Zensuswerte bekannt sind, dient der Zensus-Mittelwert als „Wahrheit“ zur Fehlerberechnung.
        \end{itemize}
        \textbf{Prädiktoren:} Alter, Geschlecht, Urbanität, Lesefähigkeit sowie Haushaltsmerkmale (PKW, Warmwasser, Wandmaterial).
    }

    \block{3. Modellstabilität (Path-Plots)}{
        Ein technisches Highlight ist die Stabilität der Regressionskoeffizienten über die 200 Simulationsläufe hinweg.
        \begin{tikzfigure}
            % Ersetze 'path_plot.png' durch dein Bild aus dem R-Code
            % \includegraphics[width=0.4\textwidth]{path_plot.png}
            \small (Hier: Grafik zur Koeffizienten-Stabilität einfügen)
        \end{tikzfigure}
        \textbf{Erkenntnis:} Trotz kleiner Fallzahlen zeigen die BHF-Koeffizienten eine sehr hohe Konsistenz, was für die Robustheit des Modells spricht.
    }

    \column{0.5}
    \block{4. Hauptergebnis: MSE-Vergleich}
    {
        Die Modellierung führt zu einer massiven Reduktion des Schätzfehlers im Vergleich zum direkten Schätzer.
        \begin{tikzfigure}
            % Ersetze 'karten.png' durch deinen Karten-Vergleich
            % \includegraphics[width=0.45\textwidth]{karten_vergleich.png}
        \end{tikzfigure}
        \textbf{Visualisierung:} Drei Karten von Bolivien (Direct vs. FH vs. BHF) zeigen die räumliche Verteilung des Mean MSE. \\
        \textbf{MSE-Distribution:}
        \begin{tikzfigure}
            % \includegraphics[width=0.4\textwidth]{violin_mse.png}
        \end{tikzfigure}
        Der Violin-Plot zeigt, dass BHF die geringste Fehlervarianz über alle Domänen aufweist.
    }
\end{columns}

%%%%%%%%%%%%%%%%%%%%%%%%%%%%%%%%%%%%%%%%%%%%%%%%%%%%%%%%%%%%%%

\begin{columns}
    \column{0.6}
    \block{5. Fazit}{
        \begin{enumerate}
            \item \textbf{BHF-Überlegenheit:} Das Unit-Level Modell (BHF) nutzt die Daten am effizientesten und liefert die stabilsten Ergebnisse bei $n=20$.
            \item \textbf{MSE-Reduktion:} Beide SAE-Modelle schlagen den direkten Schätzer in Bezug auf Präzision und Bias deutlich.
            \item \textbf{Praxisrelevanz:} Die Einbeziehung von Zensus-Hilfsvariablen ist essenziell für verlässliche regionale Statistiken in Bolivien.
        \end{enumerate}
    }
    
    \column{0.4}
    \block{Modellparameter (Beispiel)}{
    \LARGE
          \begin{tabular}{l|c|c}
            \toprule
            Variable & Schätzer (BHF) & Signifikanz \\
            \midrule
            Alter & -0.045 & *** \\
            Urbanität & 1.230 & *** \\
            Lesefähigkeit & 2.105 & *** \\
            \bottomrule
        \end{tabular}
    }
\end{columns}

%%%%%%%%%%%%%%%%%%%%%%%%%%%%%%%%%%%%%%%%%%%%%%%%%%%%%%%%%%%%%%%
\block{Referenzen}{
    \small
    Battese, G. E., Harter, R. M., & Fuller, W. A. (1988). An Error-Components Model for Prediction of County Crop Areas Using Survey and Satellite Data. \textit{JASA}. \\
    Fay, R. E., & Herriot, R. A. (1979). Estimates of Income for Small Places: An Application of James-Stein Procedures to Census Data. \textit{JASA}.
}

%%%%%%%%%%%%%%%%%%%%%%%%%%%%%%%%%%%%%%%%%%%%%%%%%%%%%%%%%%%

\block{}{
    \begin{minipage}[t]{0.3\textwidth}
        \textbf{Kontakt:}\\
        [Eure Namen]\\
        Universität Bamberg\\
        [Email]
    \end{minipage}
    \hfill
    \begin{minipage}[t]{0.3\textwidth}
        \textbf{Software:}\\
        R-Packages: \texttt{emdi}, \texttt{saeTrafo}, \texttt{tidyverse}
    \end{minipage}
}

\end{document}